\documentclass{article}
\usepackage[a4paper, margin=1in]{geometry}
\usepackage{ctex}
\usepackage{amsmath}
\usepackage{amssymb}
\usepackage{graphicx}
\usepackage{longtable}
\title{高级算法设计与分析——蔡老师考试题目}
\author{}
\date{}
\usepackage{listings}
\usepackage{xcolor}
\lstset{
  language=Pascal,              
  basicstyle=\ttfamily\small,   
  keywordstyle=\color{blue},     
  commentstyle=\color{gray},     
  stringstyle=\color{orange},   
  numbers=left,                
  numberstyle=\tiny\color{gray},
  stepnumber=1,                  
  numbersep=8pt,
  frame=single,            
  breaklines=true,            
  backgroundcolor=\color{yellow!5},
  tabsize=2,
  captionpos=b                 
}

\usepackage[colorlinks=true, allcolors=blue]{hyperref}
\usepackage{float}

\begin{document}
% \maketitle
\vspace{-9em}
% 本文参考:

% [1]~~ \url{https://zhuanlan.zhihu.com/p/401665397}

% [2]~~ \url{https://zhuanlan.zhihu.com/p/634879711}
\section*{图着色问题}
\textbf{问题背景:}
图着色问题(Graph Coloring Problem)是一种经典的组合优化问题,通常定义为:给定一个无向图$G=(V,E)$和整数$k$,求一种将顶点$V$染色的方案,使得相邻的两个顶点不能使用相同的颜色,并且使用的颜色数量不超过$k$。确定是否存在这样一种方案属于NP完全问题。




\textbf{编码成Partial MaxSAT:}
\begin{itemize}
    \item 变量定义:对于每个顶点$v_i \in V$和每个颜色$j \in \{1,2,\dots,k\}$,定义布尔变量$x_{i,j}$表示顶点$v_i$是否被赋予颜色$j$。
    \[
    x_{i,j} = 
    \begin{cases}
    1, & \text{如果顶点$v_i$被赋予颜色$j$}\\[4pt]
    0, & \text{否则}
    \end{cases}
    \]

    \item 硬约束(Hard Clauses):
    \begin{enumerate}
        \item 每个顶点至少要被赋予一种颜色:
        \[
        \bigvee_{j=1}^{k} x_{i,j},\quad \forall v_i \in V
        \]

        \item 每个顶点最多被赋予一种颜色:
        \[
        \neg x_{i,j} \vee \neg x_{i,l},\quad \forall v_i \in V, \forall j,l \in \{1,\dots,k\}, j<l
        \]

        \item 相邻顶点不能使用相同颜色:
        \[
        \neg x_{i,j} \vee \neg x_{p,j},\quad \forall (v_i,v_p)\in E, \forall j\in\{1,\dots,k\}
        \]
    \end{enumerate}

    \item 软约束(Soft Clauses):
    
    希望使用尽可能少的颜色,定义软约束来最小化所用颜色数目。可引入变量$y_j$表示颜色$j$是否被使用,并添加软约束使得使用颜色的数量最少:
    \[
    \neg y_j \vee x_{i,j},\quad \forall v_i\in V, \forall j\in\{1,\dots,k\}
    \]
    
    并且在目标函数中添加如下软约束以最小化颜色使用:
    \[
    \text{Minimize: } \sum_{j=1}^{k} y_j
    \]

\end{itemize}



\newpage
\section*{最大团问题}
\textbf{问题背景:}
最大团问题(Maximum Clique Problem)是图论中的一个经典NP难问题。给定一个无向图$G=(V,E)$,一个\textbf{团(clique)}是指图中一个完全连接的顶点子集,即子集中的每对顶点之间都存在边。最大团问题的目标是找到一个最大的团(包含顶点数目最多的团)。该问题广泛应用于社交网络分析、生物信息学、通信网络等领域。


\textbf{编码成Partial MaxSAT:}
\begin{itemize}
    \item 变量定义:对于每个顶点$v_i \in V$,定义布尔变量$x_i$,表示顶点$v_i$是否属于所选的团:
    \[
    x_i = 
    \begin{cases}
    1, & \text{若顶点$v_i$属于团中} \\[4pt]
    0, & \text{否则}
    \end{cases}
    \]

    \item 硬约束(Hard Clauses):团要求所有被选中的顶点两两之间必须有边相连。因此,对于任意一对不相邻的顶点$(v_i,v_j)\notin E$,添加硬约束:
    \[
    \neg x_i \vee \neg x_j,\quad \forall (v_i,v_j)\notin E
    \]

    该约束确保选中的顶点集必定形成一个团。

    \item 软约束(Soft Clauses):目标是最大化团的规模,因此对每个顶点$v_i\in V$,添加软约束,以鼓励选择更多顶点进入团中:
    \[
    (x_i), \quad \forall v_i \in V
    \]

    最终,Partial MaxSAT求解器的目标是满足所有硬约束,同时最大化被满足的软约束数量,即选择更多顶点以形成最大团。
    
    因此,该Partial MaxSAT问题的目标函数可表示为:
    \[
    \text{Maximize: } \sum_{v_i\in V} x_i
    \]

\end{itemize}

\newpage
\section*{最大割问题}
\textbf{问题背景:}
最大割问题(Maximum Cut Problem)是组合优化领域中的经典问题之一。给定一个无向加权图$G=(V,E,w)$,其中$w:E\rightarrow\mathbb{R}^+$是边的权重函数。最大割问题的目标是将顶点集合$V$划分为两个不相交的子集,使得连接两个子集的边的权重之和最大化。最大割问题属于NP难问题,在统计物理、组合优化和电路设计等领域有重要的应用。

形式化地,最大割问题可表述为:
\[
\max_{S\subseteq V} \sum_{\substack{(u,v)\in E \\ u\in S, v\in V\setminus S}} w(u,v)
\]

\textbf{编码成Partial MaxSAT:}

\begin{itemize}
    \item 变量定义:对每个顶点$v_i\in V$,定义布尔变量$x_i$表示该顶点所在的分区:
    \[
    x_i=
    \begin{cases}
        1,& \text{若顶点$v_i$属于第一个子集$S$}\\[4pt]
        0,& \text{若顶点$v_i$属于第二个子集$V\setminus S$}
    \end{cases}
    \]

    \item 硬约束(Hard Clauses):最大割问题本质上并无强制约束需要满足,因此不需要添加硬约束。

    \item 硬约束(Soft Clauses):对于图中的每一条边$(v_i,v_j)\in E$,添加以下两个子句:

    \begin{enumerate}
        \item 子句一:$(x_i \vee x_j)$
        \item 子句二:$(\neg x_i \vee \neg x_j)$
    \end{enumerate}
    
    \item 与最大割问题的关联目标:
    \begin{itemize}
        \item 每条被切割的边(连接不同集合的顶点对),为Max-SAT问题贡献2个满足的子句。
        \item 每条未被切割的边(连接相同集合的顶点对),为Max-SAT问题贡献1个满足的子句。
    \end{itemize}

    \end{itemize}


\newpage
\section*{顶点覆盖问题}
\textbf{问题背景:}
顶点覆盖问题(Vertex Cover Problem)是组合优化领域中一个典型的NP完全问题。给定一个无向图$G=(V,E)$,顶点覆盖是顶点集的一个子集$C \subseteq V$,满足对图中每条边$(u,v) \in E$,至少有一个端点属于$C$。顶点覆盖问题的目标是在图中找到一个最小的顶点覆盖集合,即包含最少顶点数的覆盖集合。该问题在网络安全、资源配置、任务调度等领域具有广泛的应用。

形式化地,顶点覆盖问题可表述为:
\[
\min_{C \subseteq V} |C|,\quad \text{满足:}\forall (u,v)\in E,\ u \in C\vee v \in C
\]


\textbf{编码成Partial MaxSAT:}
\begin{itemize}
    \item 变量定义:对每个顶点$v_i\in V$,定义布尔变量$x_i$表示顶点$v_i$是否属于顶点覆盖集合$C$:
    \[
    x_i =
    \begin{cases}
    1, & \text{若$v_i\in C$}\\[4pt]
    0, & \text{否则}
    \end{cases}
    \]

    \item 硬约束(Hard Clauses):对于图中的每一条边$(v_i,v_j)\in E$,必须至少有一个端点被选择进入集合$C$,因此添加硬约束:
    \[
    (x_i \vee x_j),\quad \forall (v_i,v_j)\in E
    \]

    这确保集合$C$为一个合法的顶点覆盖。

    \item 软约束(Soft Clauses):目标是找到最小的顶点覆盖集合,因此需要对每个顶点$v_i\in V$加入软约束以最小化被选中的顶点数目:
    \[
    (\neg x_i),\quad \forall v_i\in V
    \]

    最终,Partial MaxSAT求解器的目标是满足所有硬约束,并同时最大化被满足的软约束(即选择更少的顶点)。

    目标函数可表述为:
    \[
    \text{Minimize: } \sum_{v_i\in V} x_i
    \]

\end{itemize}


\newpage
\section*{最小支配集问题}
\textbf{问题背景:}
最小支配集问题(Minimum Dominating Set Problem)是组合优化中的经典NP完全问题之一。给定一个无向图$G=(V,E)$,一个顶点子集$D \subseteq V$称为支配集(Dominating Set),当且仅当对于图中每个顶点$v\in V$,要么$v\in D$,要么$v$邻接的至少一个顶点属于集合$D$。最小支配集问题的目标是在图中找到一个顶点数目最少的支配集。该问题广泛应用于传感器网络、设施选址、路由和社会网络分析等场景。

形式化描述为:
\[
\min_{D \subseteq V} |D|,\quad\text{满足:}\forall v\in V,\quad v\in D \vee (\exists u\in D, (u,v)\in E)
\]

\textbf{编码成Partial MaxSAT:}
将最小支配集问题编码为Partial MaxSAT问题的过程如下:

\begin{itemize}
    \item 变量定义:对于每个顶点$v_i\in V$,定义布尔变量$x_i$表示该顶点是否属于支配集$D$:
    \[
    x_i =
    \begin{cases}
        1, & \text{若$v_i\in D$}\\[4pt]
        0, & \text{否则}
    \end{cases}
    \]

    \item 硬约束(Hard Clauses):为确保支配集定义满足,每个顶点$v_i\in V$都必须被覆盖,即每个顶点或其邻居中至少有一个被选入集合。因此,对于每个顶点$v_i\in V$添加如下硬约束:
    \[
    \left(x_i \vee \bigvee_{(v_i,v_j)\in E} x_j\right),\quad\forall v_i\in V
    \]

    该约束保证了每个顶点都被支配集覆盖。

    \item 软约束(Soft Clauses):目标是最小化支配集的规模。因此对每个顶点$v_i\in V$,添加如下软约束,鼓励选取更少顶点:
    \[
    (\neg x_i),\quad\forall v_i\in V
    \]

    最终,Partial MaxSAT求解器的目标是满足所有硬约束,同时最大化满足的软约束,即使支配集规模尽可能小。

    目标函数表达式可记为:
    \[
    \text{Minimize: } \sum_{v_i\in V} x_i
    \]

\end{itemize}

\newpage
\section*{最小加权顶点覆盖问题}
\textbf{问题背景:}
最小加权顶点覆盖问题(Minimum Weighted Vertex Cover Problem)是顶点覆盖问题的加权版本。给定一个无向图$G=(V,E)$,以及顶点权重函数$w: V \rightarrow \mathbb{R}^{+}$,加权顶点覆盖问题的目标是找到一个顶点覆盖集合$C \subseteq V$,使得集合中顶点权重之和最小。即要求满足每条边至少有一个端点位于覆盖集合中的同时,最小化覆盖集合的总权重。

形式化地,问题可描述为:
\[
\min_{C \subseteq V} \sum_{v_i \in C} w(v_i),\quad\text{满足:}\forall (u,v)\in E,\quad u\in C\vee v\in C
\]

该问题在通信网络设计、资源分配、设施选址等实际应用领域均具有重要的现实意义。
\textbf{编码成Partial MaxSAT:}
\begin{itemize}
    \item 变量定义:对每个顶点$v_i\in V$,定义布尔变量$x_i$表示该顶点是否属于顶点覆盖集合$C$:
    \[
    x_i=
    \begin{cases}
    1, & \text{若$v_i\in C$}\\[4pt]
    0, & \text{否则}
    \end{cases}
    \]

    \item 硬约束(Hard Clauses):对于图中每条边$(v_i,v_j)\in E$,至少有一个端点必须属于覆盖集合$C$,因此添加如下硬约束:
    \[
    (x_i \vee x_j),\quad\forall (v_i,v_j)\in E
    \]

    \item 软约束(Soft Clauses):目标是最小化覆盖集合中顶点的权重之和,因此,对于每个顶点$v_i\in V$,添加权重为$w(v_i)$的如下软约束:
    \[
    (\neg x_i): w(v_i),\quad\forall v_i\in V
    \]

    Partial MaxSAT求解器的目标为满足所有硬约束的同时,尽量满足更多的软约束,从而最小化覆盖集合的权重。

    因此,目标函数的表达式为:
    \[
    \text{Minimize: } \sum_{v_i\in V} w(v_i)\cdot x_i
    \]

\end{itemize}
\newpage
\section*{最小集合覆盖问题}
\textbf{问题背景:}
最小集合覆盖问题(Minimum Set Cover Problem)是经典的组合优化问题之一,属于NP完全问题。问题定义如下:给定一个全集$U=\{u_1,u_2,\dots,u_n\}$和一个集合族$\mathcal{S}=\{S_1,S_2,\dots,S_m\}$,其中每个集合$S_j \subseteq U$,且$\bigcup_{j=1}^m S_j = U$,最小集合覆盖问题的目标是在$\mathcal{S}$中找到尽可能少的集合,使它们的并集能够覆盖整个全集$U$。该问题在资源配置、设施选址、网络设计、任务调度等领域均有广泛应用。

问题可形式化为:
\[
\min_{\mathcal{C}\subseteq \mathcal{S}} |\mathcal{C}|,\quad \text{满足:} \bigcup_{S_j \in \mathcal{C}} S_j = U
\]

\textbf{编码成Partial MaxSAT:}
\begin{itemize}
    \item 变量定义:对集合族$\mathcal{S}$中的每个集合$S_j$定义布尔变量$x_j$,表示是否选取该集合:
    \[
    x_j = 
    \begin{cases}
        1, & \text{若集合$S_j$被选中}\\[4pt]
        0, & \text{否则}
    \end{cases}
    \]

    \item 硬约束(Hard Clauses):为了保证全集中每个元素都被覆盖,对于全集$U$中的每个元素$u_i$,至少存在一个包含它的集合$S_j$被选取。因此,对于每个元素$u_i \in U$,添加硬约束:
    \[
    \bigvee_{j:u_i \in S_j} x_j,\quad\forall u_i\in U
    \]

    \item 软约束(Soft Clauses):目标是最小化所选集合的数目,因此对每个集合$S_j\in\mathcal{S}$添加如下软约束,以最小化所选集合数量:
    \[
    (\neg x_j),\quad\forall S_j\in \mathcal{S}
    \]

    Partial MaxSAT求解器的目标是满足所有硬约束并同时最大化被满足的软约束,从而实现最小化集合覆盖规模的目标。

    因此,目标函数的表达式为:
    \[
    \text{Minimize: } \sum_{S_j \in \mathcal{S}} x_j
    \]

\end{itemize}
\newpage
\section*{算法:改进的局部搜索顶点覆盖}
\begin{lstlisting}[caption={ImprovedLocalSearchVertexCover},label={lst:localsearch}]
Algorithm: ImprovedLocalSearchVertexCover(G, k, MaxIterations)
Input:
    G = (V, E)              // 给定的无向图
    k                       // 需要选取的顶点数
    MaxIterations           // 最大迭代次数限制
Output:
    BestCoverSet            // 覆盖尽可能多边的k个顶点集合

Initialize:
    CurrentSet ← 选取图G中度数最高的k个顶点
    BestCoverSet ← CurrentSet
    BestCoveredEdges ← CountCoveredEdges(G, BestCoverSet)

for iteration from 1 to MaxIterations do:
    improvement ← False

    for each vertex u in CurrentSet do:
        for each vertex v in (V - CurrentSet) do:
            NewSet ← CurrentSet - {u} + {v}
            CoveredEdges ← CountCoveredEdges(G, NewSet)

            if CoveredEdges > BestCoveredEdges then:
                CurrentSet ← NewSet
                BestCoveredEdges ← CoveredEdges
                BestCoverSet ← NewSet
                improvement ← True
                break  // 一旦找到改善的解,立即跳出内层循环

        if improvement then
            break  // 改善后重新开始外层循环

    if not improvement then:
        break  // 达到局部最优,终止搜索

return BestCoverSet

Function CountCoveredEdges(G, VertexSet):
    Covered ← 0
    for each edge (u, v) in E do:
        if u in VertexSet or v in VertexSet then:
            Covered ← Covered + 1
    return Covered
\end{lstlisting}
\newpage
\section*{局部最优解的改进策略}

下面介绍三种常用的跳出局部最优的策略,每种策略先给出核心思想,再说明如何在算法里实现。

\begin{description}
  \item[策略一:随机重启 (Random Restarts)]
  
  \textbf{思想:} 当算法陷入某个局部最优时,不立即停止,而是记录当前的最佳解,然后重新生成一个完全不同的初始解,再次运行局部搜索。  
  
  \textbf{实现:} 将主算法重复执行 \(N\) 次,每次从一个新的随机(或基于贪心的)初始解开始;最后返回这 \(N\) 次运行中找到的最优解。

  \item[策略二:允许“坏”移动 (Simulated Annealing)]
  
  \textbf{思想:} 模拟物理退火过程。在搜索初期,以一定概率接受一次“坏”的移动(即增益 \(Gain < 0\)),以跳出当前“山谷”并探索其他区域;随着迭代,接受坏移动的概率逐渐降低。  
  
  \textbf{实现:} 对每次候选交换计算增益 \(Gain\):  
  \begin{itemize}
    \item 若 \(Gain > 0\),则始终接受该移动;  
    \item 若 \(Gain \le 0\),则以概率 \(p = \exp\bigl(Gain / T\bigr)\) 接受,  
  \end{itemize}
  其中 \(T\) 为“温度”参数,随迭代次数按照预定退火表降温。

  \item[策略三:禁忌搜索 (Tabu Search)]
  
  \textbf{思想:} 为防止在几个解间来回振荡,引入“禁忌列表”(Tabu List),禁止近期的反向移动,以强制算法探索新的邻域。 
  
  \textbf{实现:} 每当将顶点 \(v_{\text{out}}\) 从当前解移出时,将其加入禁忌列表;在接下来的若干迭代中,禁止将这些顶点重新加入解集中。列表可设固定长度,超过后按 FIFO 规则释放最旧元素。
\end{description}
\begin{description}
  \item[策略四:迭代局部搜索 (Iterated Local Search, ILS)]
  
  \textbf{思想:} 在当前局部最优解的基础上,施加一次“扰动”生成一个新解,再对新解进行局部搜索,不断在扰动和精炼之间交替,以发现更优解。  
  
  \textbf{实现:} 每次从当前最优解执行轻度扰动(如随机交换若干顶点),得到候选解;然后对该解运行完整的局部搜索;重复 \(N\) 次,保留迭代过程中出现的全局最优。

  \item[策略五:贪婪随机自适应搜索 (GRASP)]
  
  \textbf{思想:} 将构造和局部搜索两阶段结合:在构造阶段使用“贪婪+随机”策略生成初始解,再对其做局部搜索;多次重复,最终取最优。  
  
  \textbf{实现:}  
  \begin{enumerate}
    \item \emph{构造阶段:} 根据增益值构造候选列表(RCL),从中随机挑选元素加入解;  
    \item \emph{局部搜索阶段:} 对构造解执行标准局部搜索;  
    \item 重复上述两阶段 \(N\) 次,保留最优解。
  \end{enumerate}

  \item[策略六:变邻域搜索 (Variable Neighborhood Search, VNS)]
  
  \textbf{思想:} 利用一系列不同规模或类型的邻域算子,从小邻域到大邻域逐步扩展,以跳出当前局部最优。  
  
  \textbf{实现:}  
  \begin{enumerate}
    \item 依次定义多个邻域结构(如单顶点交换、双顶点交换、三顶点重组等);  
    \item 在第 \(k\) 个邻域内进行扰动并局部搜索;若找到更优解则回到第一个邻域,否则进入下一个邻域;  
    \item 当所有邻域都尝试完毕后,重新从最优解开始。
  \end{enumerate}

  \item[策略七:基于学习的自适应搜索 (Adaptive Memory Programming)]
  
  \textbf{思想:} 利用历史搜索信息(如成功移动模式、频繁交换对等)来指导当前搜索,动态调整扰动或选择算子的概率。  
  
  \textbf{实现:}  
  \begin{itemize}
    \item 对每种移动操作(如某种类型的顶点交换)维护一个评分;  
    \item 每轮根据评分分布随机或贪心地选取扰动算子;  
    \item 搜索过程中实时更新评分,强化有效算子的优先级。
  \end{itemize}
\end{description}
\end{document}